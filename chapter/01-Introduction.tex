\chapter{Introduction}

\TODO{The softer part of the introduction}

This dissertation builds on existing work on Accelerate, and also array languages generally. Both prior to and during my contribution, numerous other researchers have also made many contributions. The research which I have contributed to is listed below. I explicitly state which work is my own:

\begin{itemize}
\item We have developed a new parallel structure in addition to the parallel array. Irregular arrays sequences offer a single level of nesting with greater modularity and controlled memory usage (Chapter~\ref{chap:motivation}).
\item I developed a novel extension to Blelloch's flattening transform~\cite{Blelloch:compiling1988,Blelloch:nesl1995} that identifies regular (sub)computations and transforms them according to different rules than irregular computations. By doing this I was able to take advantage of the greater efficiency afforded by regular computations, but fall back to less efficient methods when computations are irregular (Chapter~\ref{chap:theory}).
\item I implemented a system for treating GPU memory as a cache for GPU computations. A programmer using Accelerate with this feature does not have to be as concerned about their programs working memory fitting into the, limited, memory of the GPU (Section~\ref{sec:gpu-gc}.
\item I developed an FFI for Accelerate. This was the first, to my knowledge, FFI for an embedded language. The technique for doing this is applicable to all embedded languages with multiple execution backends. In Accelerate, it allows for highly optimised low-level libraries to be called from within a high-level Accelerate program and for Accelerate programs to be called from CUDA C programs (Section~\ref{sec:foreign}).
\end{itemize}

Before we describe any of this work, however, we need to first introduce the background and context of our research in Chapter~\ref{chap:background}.
