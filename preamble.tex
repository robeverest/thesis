\documentclass[11pt,a4paper,twoside,dvipsnames,final]{book}

\usepackage{hyperref}
\hypersetup{%
    pdftitle={\thesistitle},
    pdfauthor={Robert Clifton-Everest},
    hidelinks,
    breaklinks=true
}
\usepackage{a4wide}
\usepackage[notext]{stix}
\usepackage{amsfonts}
\usepackage{amsmath}
\usepackage{amssymb}
\usepackage{amsthm}
\usepackage{bibentry}
\usepackage{bigfoot}
\usepackage{booktabs}
\usepackage{breakcites}
\usepackage{caption}
\usepackage[splitrule,bottom]{footmisc}
\usepackage{framed}
\usepackage{graphicx}
\usepackage{ifdraft}
\usepackage[final]{listings}
\usepackage{marginnote}
\usepackage{mathpartir}
\usepackage{multicol}
\usepackage[sort&compress,nonamebreak,numbers]{natbib}
\usepackage{pdf14}
\usepackage{pdflscape}
\usepackage{pdfpages}
\usepackage{semantic}
\usepackage{setspace}
\usepackage{soul}     % \hl{} command for highlighting
\usepackage{stmaryrd}
\usepackage{tabularx}
\usepackage{textcomp}
\usepackage[width={autoauto}]{thumbs}
\usepackage{upgreek}
\usepackage{upquote}
\usepackage{xcolor}
% \usepackage[normalem]{ulem}

\onehalfspacing

% \DeclareCaptionFormat{captionWithLine}{\hspace{-8.5pt}\rule{\textwidth}{0.4pt}\\#1#2#3}
\captionsetup{margin=8.5pt,labelfont={bf}}

% deprecated
% \usepackage{haskell}
% \usepackage{code}

\newcommand{\note}[3]{%
  \sethlcolor{#1}%
  \hl{\emph{\textbf{#2} #3}}%
  \sethlcolor{yellow}% reset
}

% Adding the colour percentages inline seems to mess up soul
\colorlet{LightRed}{red!20}

\newcommand{\TODO}[1]           {\note{LightRed}{TODO:}{#1}}
\newcommand{\Exam}[1]           {\note{LightRed}{EXAMINER:}{#1}}
\newcommand{\finaltodo}[1]      {\note{LightRed}{TODO FINAL:}{#1}}
\newcommand{\REF}               {\note{LightRed}{REF}}
\newcommand{\CITE}              {\note{LightRed}{CITE}}

\newcommand{\NOTE}[1]{\marginnote{%
  \sethlcolor{LightRed}%
  \hl{\emph{#1}}%
  \sethlcolor{yellow}%
}}

\interfootnotelinepenalty=10000

\newcommand{\Vector}{\texttt{Vector}}
\newcommand{\lift}[1]{{#1}^{\mathrm{chunk}}}
\newcommand{\vectorised}[1]{\mathcal{V}\llbracket #1 \rrbracket }
\newcommand{\lifted}[2]{\mathcal{L}\llbracket #1 \rrbracket_{#2} }
\newcommand{\flattened}[2]{\mathcal{F}( #1,\ #2 ) }
\newcommand{\bnfdef}{\ensuremath{::=}}
\newcommand{\alt}{\ensuremath{\;|\;}}

\newcommand{\nesl}{\textsc{Nesl}}
\newcommand{\vcode}{\textsc{Vcode}}
\newcommand{\plisp}{\textsc{Paralation Lisp}}
\newcommand{\ndp}{\textsc{Acc}$_{\textsl{NDP}}$}

% listings
\lstloadlanguages{Haskell}

\lstset{%
    frame=none,
    % rulecolor={\color[gray]{0.7}},
    numbers=none,
    basicstyle=\scriptsize\ttfamily,
    % basicstyle=\footnotesize\ttfamily,
    % basicstyle=\scriptsize\ttfamily,
    numberstyle=\color{Gray}\tiny\it,
    commentstyle=\color{MidnightBlue}\itshape,
    stringstyle=\color{Maroon},
    keywordstyle=[1],
    keywordstyle=[2]\color{ForestGreen},
    keywordstyle=[3]\color{Bittersweet},
    keywordstyle=[4]\color{RoyalPurple},
    captionpos=b,
    aboveskip=2\medskipamount,
    xleftmargin=0.5\parindent,
    xrightmargin=0.5\parindent,
    % flexiblecolumns=false,
    basewidth={0.5em,0.45em},           % default {0.6,0.45}
    escapechar={\%},
    texcl=true                          % tex comment lines
}

\lstdefinestyle{haskell}{%
    language=Haskell,
    upquote=true,
    deletekeywords={case,class,data,default,deriving,do,in,instance,let,of,type,where,newtype},
    morekeywords={[2]class,data,default,deriving,family,instance,newtype,type,where},
    morekeywords={[3]in,let,case,of,do},
    morecomment=[s][\color{Orange}]{`}{`},
    literate=
        {\\}{{$\uplambda$}}1
        {\\\\}{{\char`\\\char`\\}}1
        {>->}{>->}3
        {>>=}{>>=}3
        {->}{{$\rightarrow$}}2
        {>=}{{$\geq$}}2
        {<-}{{$\leftarrow$}}2
        {<=}{{$\leq$}}2
        {=>}{{$\Rightarrow$}}2
        {|}{{$\mid$}}1
        {~}{{$\sim$}}1
        {forall}{{$\forall$}}1
        {exists}{{$\exists$}}1
        {...}{{$\dots$}}3
        {.+}{{$\oplus$}}1
%       {`}{{\`{}}}1
%       {\ .}{{$\circ$}}2
%       {\ .\ }{{$\circ$}}2
}

\lstdefinestyle{inline}{%
    style=haskell,
    basicstyle=\ttfamily,
    % basicstyle=\footnotesize\ttfamily,
    % keywordstyle=[1],
    % keywordstyle=[2],
    % keywordstyle=[3],
    % keywordstyle=[4],
    commentstyle=\itshape,
    literate=
        {\\}{{$\lambda$}}1
        {\\\\}{{\char`\\\char`\\}}1
        {>->}{>->}3
        {>>=}{>>=}3
        {->}{{$\rightarrow$\space}}3    % include forced space
        {>=}{{$\geq$}}2
        {<-}{{$\leftarrow$}}2
        {<=}{{$\leq$}}2
        {=>}{{$\Rightarrow$\space}}3
        {|}{{$\mid$}}1
%        {~}{{$\sim$}}1
        {forall}{{$\forall$}}1
        {exists}{{$\exists$}}1
        {...}{{$\cdots$}}3
        {~}{{$\sim$}}1
        {++}{{$\doubleplus$}}1
        {.+}{{$\oplus$}}1
        %
        % TLM: Rob, see above.
        % {Flat}{{$\mathcal{F}$}}2
        % {Lift}{{$\mathcal{L}$}}2
        % {Gamma}{{$\Gamma$}}1
 }

% RCE: Trev, I've moved this into their own style like you suggested. I think it's best to keep it as a listing rather than using math. It's close enough to haskell that it's not worth the effort.
 \lstdefinestyle{ndp}{%
    style=haskell,
    literate=
        {\\}{{$\uplambda$}}1
        {\\\\}{{\char`\\\char`\\}}1
        {>->}{>->}3
        {>>=}{>>=}3
        {->}{{$\rightarrow$}}2
        {>=}{{$\geq$}}2
        {<-}{{$\leftarrow$}}2
        {<=}{{$\leq$}}2
        {=>}{{$\Rightarrow$}}2
        {|}{{$\mid$}}1
        {~}{{$\sim$}}1
        {forall}{{$\forall$}}1
        {exists}{{$\exists$}}1
        {...}{{$\dots$}}3
        {Norm}{{$\mathcal{N}$}}2
        {Vect}{{$\mathcal{F}$}}2
        {Vector}{{Vector}}6
        {Lift}{{$\mathcal{L}$}}1
        {[|}{{$\llbracket$}}1
        {|]}{{$\rrbracket$}}1
        {_0}{{$_0$}}1
        {_1}{{$_1$}}1
        {_2}{{$_2$}}1
        {_3}{{$_3$}}1
        {_l}{{$_l$}}1
        {_x}{{$_x$}}1
        {_n}{{$_n$}}1
        {_n+1}{{$_{n+1}$}}3
        {_new}{{$_{\textit{new}}$}}3
        {_old}{{$_{\textit{old}}$}}3
        {_norm}{{$^{\textit{norm}}$}}4
        {_flat}{{$\!^{\textit{flat}}$}}4
        {_,flat}{{$^{\textit{flat}}$}}4
        {_1_flat}{{$_1^{\textit{flat}}$}}3
        {_2_flat}{{$_2^{\textit{flat}}$}}3
        {_3_flat}{{$_3^{\textit{flat}}$}}3
        {_n_flat}{{$_n^{\textit{flat}}$}}3
        {_new_flat}{{$\!\!_{\textit{new}}^{\textit{flat}}$}}4
        {_i}{{$_i$}}1
        {_v}{{$_v$}}1
        {_A}{{$_A$}}1
        {_S}{{$_S$}}1
        {_R}{{$_R$}}1
        {_Ir}{{$_{Ir}$}}2
        {_F}{{$_F$}}1
        {_N}{{$_\mathcal{N}$}}1
        {_V}{{$_\mathcal{V}$}}1
        {_seg}{{$_{seg}$}}3
        {_NDP}{{$_{\textit{NDP}}$}}3
        {_l}{{$_{\mathcal{L}}$}}1
        {.+}{{$\oplus$}}1
        {++}{{$\doubleplus$}}1
        {::=}{{$\Coloneqq$}}3
        {Gamma}{{$\Gamma$}}1
        {|-}{{$\vdash$}}1
 }

\lstdefinestyle{ndp_inline}{%
    style=ndp,
    basicstyle=\ttfamily
 }

\lstset{style=haskell}

\newcommand{\makeatcode}{\lstMakeShortInline[style=ndp_inline]@}
\newcommand{\makeatchar}{\lstDeleteShortInline@}

\newcommand{\inl}[1]{\lstinline[style=inline]{#1}}

\lstnewenvironment{code}%
  {\lstset{style=haskell}}%
  {}

\newcommand{\optionrule}{\noindent\rule{1.0\textwidth}{0.75pt}}

\newenvironment{aside}
{\def\FrameCommand{\hspace{2em}}
    \MakeFramed {\advance\hsize-\width}\optionrule}
{\par\vskip-\smallskipamount\optionrule\endMakeFramed}

% Bibliography
\bibliographystyle{abbrvnat}
% \setcitestyle{authoryear,open={[},close={]}}
\nobibliography*

\let\Cite\cite
\renewcommand{\cite}{\nolinebreak\Cite}

% HAX!
% \makeatletter
% \makeatother
\makeatcode